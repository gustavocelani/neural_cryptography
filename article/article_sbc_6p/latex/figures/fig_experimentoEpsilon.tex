% Figura: Experimento Epsilon
\begin{figure}[H]
	\centering
		\begin{tikzpicture}
			
			% Styles
			\tikzset{
				block1/.style={draw, align=left, thick, rectangle, text width=1.8cm, minimum height=1cm},
				block2/.style={draw, align=center, thick, rectangle, text width=1.5cm, minimum height=0.5cm},
				block3/.style={draw, align=center, thick, rectangle, text width=5.5cm, minimum height=1.25cm},
			}
			
			% Layer 1 Nodes
			\node[block1] (kn) {K ótimo\\N ótimo};
			
			% Layer 2 Nodes
			\node[block2, right=1.5cm of kn] (l3)  {\textit{L} = 4};
			\node[block2, above=0.1cm of l3] (l2)  {\textit{L} = 3};
			\node[block2, above=0.1cm of l2] (l1)  {\textit{L} = 2};
			\node[block2, below=0.5cm of l3] (l9)  {\textit{L} = 49};
			\node[block2, below=0.1cm of l9] (l10) {\textit{L} = 50};
			
			% Layer 3 Nodes
			%\node[block3, right=1cm of l3] (o) {\textit{Optimal}\\\textit{Activation Function}\\\textit{Optimal}\\\textit{Learning Rule}};
			\node[block3, right=1cm of l3] (o) {Função de Ativação Ótima\\Regra de Aprendizado Ótima};
			
			% Arrows
			\draw[dotted, thick] (l3) to (l9);
			\draw[-stealth, thick] (kn.east)  -- (l1.west);
			\draw[-stealth, thick] (kn.east)  -- (l2.west);
			\draw[-stealth, thick] (kn.east)  -- (l3.west);
			\draw[-stealth, thick] (kn.east)  -- (l9.west);
			\draw[-stealth, thick] (kn.east)  -- (l10.west);
			\draw[-, thick]        (l1.east)  -- (o.west);
			\draw[-, thick]        (l2.east)  -- (o.west);
			\draw[-, thick]        (l3.east)  -- (o.west);
			\draw[-, thick]        (l9.east)  -- (o.west);
			\draw[-, thick]        (l10.east) -- (o.west);

		\end{tikzpicture}
	\caption{Experimento 3.}
	\label{fig:experimentoEpsilon}
\end{figure}