% Figura: Fluxograma da Geração de chaves RSA
\begin{figure}[H]
    \centering
    \begin{tikzpicture}
                    
        % Styles
        \tikzset{
            block/.style={draw, align=center, thick, rectangle, text width=4.3cm, minimum height=0.9cm},
            dblock/.style={draw, align=center, thick, rectangle, dashed, text width=4.3cm, minimum height=0.9cm}
        }
        
        % Nodes
        \node[block]                      (rnc) {Geração Pseudoaleatória de Número};
        \node[block, right=1.5cm of rnc]  (m)   {$\textit{n} = \textit{p} \times \textit{q}$};
        \node[block, below=1cm of m]    (phi) {$\phi_{(n)} = mmc(\textit{p}-1, \textit{q}-1)$};
        \node[block, left=1.5cm of phi]   (e)   {$1 < \textit{e}' < \phi_{(n)}$};
        \node[block, below=1cm of e]    (d)   {$\textit{d} \equiv mod(\phi_{(n)})$};
        \node[dblock, right=1.5cm of d] (k)   {\textit{PVK = [n, e]}\\\textit{PBK = [p, q, d]}};

        % Arrows
        \draw[-stealth, thick] (rnc) -- (m);
        \draw[-stealth, thick] (m)   -- (phi);
        \draw[-stealth, thick] (phi) -- (e);
        \draw[-stealth, thick] (e)   -- (d);
        \draw[-stealth, thick] (d)   -- (k);

    \end{tikzpicture}
    \caption{Diagrama de geração de chaves RSA.}
    \label{fig:RSAkeyGen}
\end{figure}