% Figura: Experimento Gamma
\begin{figure}[H]
	\centering
	\begin{tikzpicture}

		% Styles
		\tikzset{
			block2/.style={draw, align=center, thick, rectangle, text width=1cm, minimum height=0.5cm},
			block3/.style={draw, align=center, thick, rectangle, text width=3.6cm, minimum height=1.25cm},
		}

		% Nodes
		\node[thick, align=center] (c) {
			\begin{tabular}{ |c|c| }
				\hline
				\multicolumn{2}{ |c| }{$C_i [k_i, n_i]$} \\
				\hline \hline
				$k_i$ & $n_i$ \\
				\hline \hline
				1    & 1024 \\ \hline
				2    & 512  \\ \hline
				4    & 256  \\ \hline
				8    & 128  \\ \hline
				16   & 64   \\ \hline
				32   & 32   \\ \hline
				64   & 16   \\ \hline
				128  & 8    \\ \hline
				256  & 4    \\ \hline
				512  & 2    \\ \hline
				1024 & 1    \\
				\hline
			\end{tabular}
		};
		\node[block2, right=0.4cm of c]  (l)  {\textit{L = 5}};
		\node[block3, right=0.5cm of l]  (o1) {Função de Ativação B\\Regra de Aprendizado B};
		\node[block3, above=0.5cm of o1] (o2) {Função de Ativação A\\Regra de Aprendizado A};
		\node[block3, below=0.5cm of o1] (o3) {Função de Ativação C\\Regra de Aprendizado C};

		% Arrows
		\draw[-stealth, thick] (c.east) -- (l.west);
		\draw[-stealth, thick] (l.east) -- (o1.west);
		\draw[-stealth, thick] (l.east) -- (o2.west);
		\draw[-stealth, thick] (l.east) -- (o3.west);

	\end{tikzpicture}
	\caption{Experimento 2.}
	\label{fig:experimentoGamma}
\end{figure}